\documentclass[12]{article}
\usepackage[margin=1.0in]{geometry}
\usepackage[utf8]{inputenc}
\usepackage{titlesec}
\usepackage{physics}
\usepackage[version=4]{mhchem}
\usepackage{graphicx}
\usepackage{siunitx}
\usepackage{cancel}
\usepackage{amsmath}
\usepackage{textcomp}
\usepackage{gensymb}
\usepackage{natbib}
\usepackage{bm}
\usepackage{setspace}

\titleformat*{\subsection}{\normalfont\fontfamily{phv}}
  \titleformat{\subsection}[runin]{\normalfont\bfseries}{\thesubsection.}{3pt}{}
  \titleformat{\subsubsection}[runin]{\normalfont\bfseries}{\thesubsubsection.}{3pt}{}

\title{{\textsc{\Large Nuclear Power Plant Shutdown Impact on Particulate Matter and Ozone across the United States}}}
\author{\textsc{Lyssa Freese}
\\\\
Advised by Prof. Noelle Selin}

\doublespace
\begin{document}

\maketitle
\thispagestyle{empty}

\setlength{\leftskip}{1.1cm}
\setlength{\rightskip}{1.1cm}


\bigskip
\bigskip

{\textsc{Abstract.} 
The United States’ future energy mix is likely to transition from its current 20\% reliance on nuclear power. This is due to policy changes and to scheduled shutdowns of nuclear power plants that have reached the end of their lifetime. Given projected growth in demand, the base-load nature of nuclear power, and the intermittency of most renewable resources, fossil fuel-based energy will make up at least part of this deficit, impacting air quality and the associated health impacts. We develop and validate a generator-level energy grid optimization and emissions model, and couple it with the chemical transport model, GEOS-Chem, to estimate the impact that an immediate shutdown of nuclear power plants would have on concentrations of ozone and fine particulate matter (\ce{PM_{2.5}}).

When compared to a business-as-usual scenario, the nuclear shutdown scenario leads to a nationwide 49\% increase in \ce{NO_x} emissions and a 45\% increase in \ce{SO2} emissions by coal; and a 40\% increase in \ce{NO_x} emissions by natural gas. We find \ce{PM_{2.5}} increases an average of 0.4\% nationwide, with maximum local increases of up to 33\%; summertime ozone increases an average of 0.11\%, with maximum local increases of up to 16\%. Summertime ozone decreases in small regions with volatile organic compound-limited chemical regimes, including Pittsburgh, Sacramento, Bismarck and Portland. This suggests that rapidly shutting down nuclear power plants shifts the current generation mix towards dirtier fuel sources, degrading air quality nationwide. Realistically, shutdowns will occur over longer timescales during which new generation will be deployed; this work indicates the importance of prioritizing renewable deployment as new sources of generation in order to reduce these impacts. Importantly, our generator-level energy grid optimization model allows us to assess variations across the scenarios at the local scale, which occur due to a combination of local fuel sources, demand patterns, and chemical regimes.}

\bigskip
\bigskip 
\clearpage
\setcounter{page}{1}

\setlength{\leftskip}{0cm}
\setlength{\rightskip}{0cm}

\section{Introduction}
In 2018, the United States relied on nuclear power for 19\% of electricity generation; by 2050, this is expected to decrease to only 12\% \citep{eia_annual_2020}. This is due to lower natural gas prices and declining costs of renewables, as well as increasing maintenance costs of nuclear power plants \citep{davis_market_2016}. Nuclear power has played an important role in the U.S. grid over the past few decades, providing energy that has the lowest CO2 emissions (both direct and indirect); and minimal effects on human health due to air pollution \citep{markandya_electricity_2007}. 

Previous work has shown that nuclear shutdowns tend to lead to increased use of fossil fuels, as was seen in the 2012 shutdown of San Onofre’s Nuclear Plant leading to increased use of natural gas, and Tennessee Valley’s Browns Ferry and Sequoyah 1985 shutdowns which were replaced with coal use \citep{davis_market_2016,severnini_impacts_2017}. Fossil fuels have much higher emissions of fine particulate matter (\ce{PM_{2.5}}), \ce{NO_x}, \ce{SO2}, and \ce{CO2}. \ce{NO_x} and \ce{SO2} are precursors for both ozone and \ce{PM_{2.5}}, both of which are harmful to human health. Therefore, the energy transition that will occur over the next few decades is critical to understand, and currently there is a lack of literature looking into future emission and health impacts of reducing the role of nuclear power in the U.S. energy market. 

Previous work focuses on the nation-wide health and global climate impact of U.S. energy transitions, particularly the current trend of moving from coal to natural gas \citep{lueken_climate_2016, zhang_climate_2016}, as well as the use of renewables such as wind and solar \citep{millstein_climate_2017}. These studies find \ce{SO2} and \ce{NO_x} reductions from 2016 baselines of up to 90\% and 60\%, respectively, across the United States, with a range of \$20-50 billion reductions in damages to human health \citep{lueken_climate_2016}. 

However, there are two general gaps with the current approaches to these assessments. The first is that there is little work looking at the future of nuclear power, specifically shutdowns, and what this will mean for emissions across the United States. The second is that existing work tends to focus on the national level, providing health assessments and emissions changes in national terms. Given the importance of locality to specific plants and their emissions and the fact that local state and city authorities can play an important role in energy decision-making, we find it important to think about this issue in a way that can be assessed not only nationally, but at a regional and local level as well. 


\section{Methods}

\begin{figure}[!htb]
    \centering 
    \includegraphics[scale = .5]{US_EGO_flow.png}
    \caption{US-EGO model workflow.}
    \label{fig:my_label}
\end{figure}

\subsection{US Energy Grid Optimization Model (US-EGO).}
The energy grid optimization model is a generator-level cost optimization tool, initially developed by Alan Jenn at U.C. Davis for national level assessments. A student in the Aero-Astro department took this initial framework and build a simple model for the U.S., which I assisted in developing.  

US-EGO takes all energy generating units (EGU's) across the United States, their capacity, emission factors (for \ce{CO2}, \ce{SO2}, and \ce{NOx}), and their costs for the year 2016, all of which are based on the EPA's National Electric Energy Data System (NEEDS) model v.5.16 \citep{epa_power_2016}. Separating the nation into NEEDS' 64 regions, we use 2016 transmission data (from the NEEDS data) to allow for transmission between each region, and 2016 loads to create demand within each region. The model then optimizes the use of EGUs such that the load matches generation at every hour in every region. The optimization runs across $t$ time periods with 1. $x_gen$ generation for generator $i$ at cost $c_i$ with $N$ total generators, and 2. $x_trans$ transmission power between regions d and o at cost $c_{o\rightarrow{}d}$. This is run for 8760 hours throughout the year, optimizing at each timestep \citep{jenn_future_2018}
\begin{equation}
    \min\limits_{x^{gen}, x^{trans}}\sum_{i=1}^{n}\sum_{t=1}^{T} x^{gen}_{i}(t)*c^{gen}_{i}(t) + \sum_{o,d}\sum_{t=1}^{T} x^{trans}_{o\rightarrow{}d}(t)*c^{trans}_{o\rightarrow{}d}(t)
\end{equation}

The model returns hourly output of generation, from which we calculate the hourly emissions of \ce{SO2} and \ce{NOx}. These hourly emissions are merged onto a 0.5\degree by 0.625\degree grid. 

In order to generate the no-nuclear scenario, we remove all nuclear power plants from the possible EGUs. In this scenario, generation cannot match the U.S. energy demand in the summertime in the ERC-REST region (eastern Texas), and therefore we close the gap by adding generators that have prohibitive costs such that they are only utilized when the optimization cannot close. These generators have zero emissions, and are there to allow for the optimization to close, and thus we assume partial blackouts in ERC-REST  without nuclear power during those time periods. This in itself is an important outcome of this work, which is that with the current grid, it is unlikely that the U.S. would be able to handle large-scale immediate shutdowns of nuclear power. We understand that realistically, new energy generating units will be developed alongside shutdowns of nuclear power plants, and it is likely much of that will be natural gas or renewables rather than coal, and future work will focus on longer-term scenarios in which new energy development is considered as well.

\subsection{Chemical Transport Model: GEOS-Chem}
We use the GEOS-Chem model \citep{} version 12.6.1 \citep{noauthor_geos-chem_2019} to simulate \ce{SO2}, \ce{NOx}, \ce{PM_{2.5}} and ozone concentrations. We use a global horizontal resolution of 4\degree x 5\degree to create boundary conditions for a nested North American run with horizontal resolution of 0.5\degree by 0.625\degree between 140\degree - 40\degree W and 10\degree - 70\degree N. We run full-chemistry in the troposphere only, with 47 vertical levels. Our spin-up is four months, and we analyze daily concentration outputs for the year of 2016. GEOS-Chem version 12 has a number of improvements that reduce the high nitrate bias seen in previous versions of GEOS-Chem \citep{walker_simulation_2012}, including 1. improved dry deposition of \ce{HNO_3} at cold temperatures \citep{jaegle_nitrogen_2018}, 2. updates to ISORROPIA, the thermodynamic model for inorganic aerosol formation to version 2.2 \citep{fountoukis_isorropia_2007}, and 3. improved treatment of heterogeneous \ce{NO_2}, \ce{NO_3}, and \ce{N_2O_5} chemistry in aerosols and clouds.

We run four GEOS-Chem simulations with different U.S. EGU emission inputs: default, e-grid, US-EGO, and no-nuclear. The default simulation takes the standard GEOS-Chem U.S. EGU emissions files, which are taken from the National Emissions Inventory (NEI) 2011 emissions scaled to the year 2013. The e-grid simulation utilizes the EPA's Emission and Generation Resource Integrated Database (e-grid) \citep{epa_emissions_2016} \ce{SO2} and \ce{NOx} emissions gridded onto a 0.5\degree by 0.625\degree grid. The US-EGO simulation uses the emissions profiles created through the US-EGO model, and the no-nuclear scenario uses emissions profiles created through the US-EGO model in a no nuclear scenario. 

\begin{figure}
    \centering
    \includegraphics[scale=0.4]{ego_nonuclear_project/Figures/plants_normal.png}
    \caption{Spatial maps of coal, nuclear and natural gas power plants across the U.S.} 
    \label{fig:plants}
\end{figure}

\section{Results and Discussion}
\subsection{Model Validation}

In order to validate the model, we compare our emissions output to that of the EPA's e-grid, and the NEI 2011 data, scaled to the 2013 values as is the default in GEOS-Chem. We compare the emissions profiles before running it through GEOS-Chem, and the resulting concentrations as obtained from these three GEOS-Chem scenarios. Additionally, we compare the annual mean GEOS-Chem concentrations of each model run to two on-the-ground observational networks: the EPA Air Quality System (AQS) monitoring data for 2016 \citep{us_epa_daily_2016} and the IMPROVE network \citep{malm_spatial_1994}. 

Our model emissions by region and fuel-type correlates reasonably well with the emissions by region and fuel-type from both the e-grid and NEI (Figure***), which indicates that the optimization does an adequate job of capturing the emissions of energy generating units across the United States in the year 2016. We find bias in x regions during x times ***. From this, we can have confidence that our model does as well at capturing emissions from the EGU sector as standard datasets do. 

Relative to observations, our model 



\subsection{Ozone and \ce{PM_{2.5}} impacts of a no-nuclear scenario}

\begin{figure}
    \centering
    \includegraphics[scale=0.4]{ego_nonuclear_project/Figures/summer_winter_national_dif.png}
    \caption{Spatial maps of summer (JJA) concentration differences between scenarios (no nuclear - normal). We see the largest differences in \ce{NO_x} and \ce{SO_2} concentrations in the Northeast, leading to large differences in \ce{PM_{2.5}} and ozone in the region as well.} 
    \label{fig:summer_winter_dif}
\end{figure}


Here we look at the differences between the no nuclear and normal scenarios in the summer, June, July and August (JJA) and winter, December, January, February seasons (DJF). 

Differences between the two scenarios in \ce{SO2} and \ce{NOx} concentrations are similar in the summer and winter, with the largest differences occuring in the eastern Midwest, Northeast and Southeast, as seen in Figure ~\ref{fig:summer_dif} and Figure ~\ref{fig:winter_dif}. We find increases in both for a no nuclear scenario when compared to our normal scenario. The regions in which concentrations increase are also regions in which there is a much larger reliance on nuclear power in the normal scenario, as well as shift to reliance on coal in the no nuclear scenario, as can be seen in Figure ~\ref{fig:plants}. \ce{NOx} concentrations increase by up to 40 ppb (a 1000\% increase) in Western Pennsylvania during the wintertime, and \ce{SO2} concentrations increase by up to 18 ppb (a 3700\% increase) along the Pennsylvania-Ohio border during wintertime. 

There are overall increases in both summertime and wintertime \ce{PM_{2.5}} concentrations across the United States when nuclear power plants are shut down, with changes of up to 10 $\mu g/m^3$ occurring in June in the eastern Missouri. Increases in \ce{PM_{2.5}} during the summer time are an average of .04 $\mu g/m^3$ greater than during the winter. Particularly, in the central-western Pennsylvania, the summer time \ce{PM_{2.5}} difference is up to 1.2 $\mu g/m^3$ greater than that in the winter. 

Summertime ozone, on the other hand, decreases largely throughout the Northeast, with slight increases in the Southeast and western Midwest regions. Maximum ozone increases occur along the South Carolina-Georgia border with increases of 1.7 ppb. Decreases are largest in the Northeast, particularly along the Pennsylvania-Ohio border, decreasing by up to 21.8 ppb. 

\subsection{Ozone Regimes}
Based on \cite{jin_evaluating_2017}, we calculate a surface ratio of formaldehyde to \ce{NO_2} (FNR), which we use to evaluate the surface ozone production regime of each grid cell. For ratios < 0.5, we assume a VOC limited regime, for ratios > 0.8, we assume a \ce{NO_x} limited regime, and in between is transitional. In VOC limited regimes, an increase in \ce{NO_x} will actually contribute to decreased levels of ozone, as VOCs are fully saturated and any additional \ce{NO_x} will contribute to suppression of ozone formation. This is what occurs throughout the Northeast and eastern Midwest regions when we increase \ce{NO_x} in the no nuclear scenario. On the other hand, in \ce{NO_x} limited regimes, such as rural areas in the Southeast and Northeast, we find increased levels of \ce{NO_x} leading to increases in ozone, as the VOCs are not fully saturated and any additional \ce{NO_x} will contribute to the formation of ozone. This ratio is highly sensitive to the \ce{NO_x} and \ce{CH_2O} concentrations in each region, and we can assess the validity of these ratios based on 1. observational comparisons, and 2. whether or not a region we determined as VOC limited does have a decrease in ozone with increases in \ce{NO_x}. As can be seen in !!!FIGURE!!!, we find that !!!!!!. 

\begin{figure}
    \centering
    \includegraphics[scale=0.4]{ego_nonuclear_project/Figures/summer_regime_national_ratio.png}
    \caption{Spatial maps of summer (JJA) FNR} 
    \label{fig:summer_FNR}
\end{figure}
\subsection{\ce{PM_{2.5}} Sensitivities}
As GEOS-Chem has a high nitrate bias in the Northest, we expect this to impact our \ce{PM_{2.5}} concentrations in both scenarios; here we discuss the implications of the bias that we previously described, and explain its role in this work.

We utilize an offline version of ISORROPIAII, which allows us to look at inorganic particulate matter formation as a function of temperature, relative humidity, nitrate, sulfate, ammonia, chlorine, potassium, magnesium and calcium \citep{fountoukis_isorropia_2007}. We use regional and seasonal mean ammonia, temperature, and relative humidity values from our normal scenario model runs, and set a range for sulfate and nitrate to obtain !!!FIGURES!!!, which provide an understanding of the sensitivity of \ce{PM_{2.5}} formation to both sulfate and nitrate concentrations. As expected, \ce{PM_{2.5}} is equally sensitive to both species. We compare this to the seasonal average \ce{PM_{2.5}}, sulfate, and nitrate from IMPROVE observations at each station, and to our normal scenario model output, interpolated to the location of each IMPROVE station. We find that the model does !!!!
\section{Future work}

\section{Acknowledgements}


\pagebreak
\bibliographystyle{apalike}
\bibliography{references.bib}

\end{document}

